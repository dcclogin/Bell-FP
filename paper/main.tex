\documentclass[manuscript,review,anonymous,timestamp,acmthm,urlbreakonhyphens,acmsmall]{acmart} %% 25 pages

\usepackage{graphicx} % Required for inserting images
\usepackage{amsmath}
\usepackage[english]{babel}
\usepackage{amsthm}
\usepackage{braket}
\usepackage{tikz}
\usepackage{tikz-cd}
\usepackage{tikz-3dplot}
\usepackage{epigraph} 
\usepackage{comment}
\usepackage[dvipsnames]{xcolor}
\usepackage{mathtools}
\usepackage{booktabs}
\usepackage{float}
\usepackage[most]{tcolorbox}
\usepackage{minted}
\setminted{mathescape, escapeinside=||}

\renewcommand\qedsymbol{$\blacksquare$}
\newcommand\flowsfroma{\mathrel{\vcenter{\hbox{\rotatebox{180}{$\leadsto$}}}}}
\newcommand\flowsfromb{\mathrel{\reflectbox{$\leadsto$}}}
\newcommand{\hc}[1]{\mintinline{haskell}{#1}}

\bibliographystyle{ACM-Reference-Format}
\citestyle{acmnumeric}

\newtheorem{remark}{Remark}

\title{Functional Pearl: \\
  Bell Inequalities for Functional Programmers}
\author{Chenchao Ding}
\orcid{0000-0000-0000-0000}
\email{cd17@iu.edu}
\author{Amr Sabry}
\orcid{0000-0002-1025-7331}
\email{sabry@iu.edu}
\affiliation{Indiana University}

\begin{abstract}
  \ldots
\end{abstract}

\begin{document}
\maketitle 

%%%%%%%%%%%%%%%%%%%%%%%%%%%%%%%%%%%%%%%%%%%%%%%%%%%%%%%%%%%%%%%%%%%%%%%
\section{Introduction}

Bell’s theorem and the associated Bell experiment form one of the most
striking separations between classical and quantum information.  In
his seminal 1964 paper, Bell~\cite{Bell1964} showed that no theory
satisfying a natural \emph{locality assumption} can reproduce all of
the correlations predicted by quantum mechanics.  This abstract
separation was soon turned into a concrete experimental challenge by
Clauser, Horne, Shimony, and Holt~\cite{CHSH1969}, who reformulated
Bell’s argument as an operational inequality (now called the CHSH
inequality) suitable for laboratory tests.  A sequence of increasingly
refined experiments, most notably those of Aspect, Grangier, and
Roger~\cite{Aspect1982} in the early 1980s, and culminating in
loophole-free tests by Hensen et al.~\cite{Hensen2015}, have since
confirmed these non-classical correlations beyond reasonable doubt.
The Nobel Prize in Physics 2022 was awarded to Alain Aspect, John
F. Clauser, and Anton Zeilinger ``for experiments with entangled
photons, establishing the violation of Bell inequalities and
pioneering quantum information science.''

The experimental setup underlying Bell’s theorem, as captured by the
CHSH inequality, can be described in simple operational terms.  We are
given two black boxes.  For the moment, we model each box as a
function of type \hc{Bool -> Bool}.  Intuitively, the input represents
an experimental setting, and the output represents the observed
outcome.  We do not know how a black box is implemented internally;
all we can do is observe its input-output behavior.  Since the black
box is represented as a function, observation amounts to function
application:

\begin{minted}{haskell}
type BB = Bool -> Bool 

observe :: BB -> Bool -> Bool
observe bb b = bb b
\end{minted}

We are interested in correlations between two such boxes.  Given a pair
of boxes and one input for each, we record whether their outputs agree
or disagree.  Agreement is assigned a score of \(+1\), and disagreement
a score of \(-1\):

\begin{minted}{haskell}
scoreRun :: (BB,BB) -> (Bool,Bool) -> Int
scoreRun (bb1,bb2) (b1,b2) =
  if observe bb1 b1 == observe bb2 b2 then 1 else -1
\end{minted}

The CHSH experiment aggregates these correlations across the four
possible input pairs.  The sign pattern is chosen so that achieving the
algebraic maximum \(4\) requires a very specific behavior: agreement on
three inputs and disagreement on the fourth.

\begin{minted}{haskell}
chsh :: (BB, BB) -> Int
chsh bbs =
      scoreRun bbs (False,False)
  +   scoreRun bbs (False,True)
  +   scoreRun bbs (True,False)
  -   scoreRun bbs (True,True)
\end{minted}

With the interface \hc{BB = Bool -> Bool}, each black box is a
deterministic Boolean function on a single input bit.  Such a function
is completely determined by its outputs on \hc{False} and \hc{True}, so
there are exactly \(2^2=4\) possible black boxes.

In this purely functional model, the CHSH score is always bounded by
\(2\).  The reason is simple: if the boxes agree on the first three
input pairs in \hc{chsh}, then they are forced to agree on the fourth
as well.  Thus the ``three agreements and one disagreement'' pattern
needed to reach \(4\) cannot occur.  The best achievable value is
\(1+1+1-1=2\), and similarly the worst is \(-2\).

This conclusion is strikingly at odds with both quantum mechanics and
experiment, which exhibit correlations attaining the larger value
\(2\sqrt{2}\).  Evidently, modeling black boxes as pure functions is
too restrictive.

At the other extreme, if we allow the two boxes to communicate during
evaluation, then the CHSH task becomes trivial.  For instance, suppose
the boxes share a mutable cell, so that the first box can write its
input and the second box can read it.  Then the second box can
deliberately enforce the winning CHSH pattern: agree on the three
positively weighted input pairs and disagree on the fourth.  This
immediately yields the algebraic maximum \(4\).

This illustrates why Bell experiments insist on a locality constraint.
The physical assumption is that the two boxes are spacelike separated:
once prepared, they are not allowed to communicate.  In
programming-language terms, this rules out direct message passing,
shared mutable state, or any other mechanism by which one box could
influence the other during evaluation.

We are therefore looking for a semantic middle ground.  We want to
enlarge the space of admissible black boxes beyond pure functions so as
to account for the observed value \(2\sqrt{2}\), while still enforcing a
meaningful notion of locality.

A natural next step is to allow black boxes to be effectful.  A box
might flip coins, consult shared randomness fixed at preparation time,
or interact with a read-only environment.  These behaviors are familiar
from programming-language semantics and can be modeled uniformly using
monads.  Accordingly, we generalize the type of black boxes to
\hc{BB m = Bool -> m Bool}:

\begin{minted}{haskell}
type BB m = Bool -> m Bool
\end{minted}

This raises a further question: what does locality mean for effectful
black boxes?  Given \hc{bb1 :: BB m} and \hc{bb2 :: BB m}, and a pair of
inputs \hc{(b1,b2)}, we can always form a joint experiment of type
\hc{m (Bool,Bool)}.  One formalization of locality is Bell’s
\emph{factorization} principle: once we condition on whatever was fixed
at preparation time, the two outputs are generated independently.  In
programming-language terms, this corresponds to combining effects via
the canonical applicative pairing:

\begin{minted}{haskell}
eval (bb1,bb2) (b1,b2) = liftA2 (,) (bb1 b1) (bb2 b2)
\end{minted}

When the effect \hc{m} is the distribution monad, this equation
specializes to Bell’s factorization condition:
\[
p(a,b \mid b_1,b_2,\lambda)
=
p(a \mid b_1,\lambda)\,p(b \mid b_2,\lambda),
\]
where mixing over \(\lambda\) corresponds to averaging over different
preparations.

Bell’s theorem implies that (under the usual assumptions, in particular
that preparation is independent of the later choice of inputs) any
factorizing model is bounded by \(2\).  To reach \(2\sqrt{2}\), one must
relax something.  Crucially, one can relax factorization \emph{without}
allowing communication: quantum correlations violate Bell factorization
while still obeying the weaker no-signalling principle.

Operationally, this means that the evaluation function \hc{eval} cannot
always be the canonical applicative pairing of the two local effects.
In the rest of the paper, we make this precise by giving an operational
semantics parameterized by an effect, and by isolating the exact point
where classical applicative factorization must be replaced by a more
delicate composition principle.

%=======================================================
\section{Locality assumptions as algebraic laws}
\label{sec:locality-laws}

The discussion in the introduction highlights three distinct regimes.
If we model each black box as a pure function, then the CHSH score is
bounded by \(2\).  If, on the other hand, we allow unrestricted
interaction between the two boxes during evaluation, then the algebraic
maximum \(4\) becomes trivial to achieve.  Quantum correlations sit
strictly between these extremes: they exceed \(2\) while still obeying a
meaningful notion of locality.

In the literature on Bell’s theorem, several formal locality principles
are commonly used to express what ``no communication'' should mean.
Although these principles are usually stated in probabilistic language,
they can be formulated more generally as algebraic constraints on an
operational semantics.  The purpose of this section is to collect these
assumptions in a uniform form and to show how each can be represented as
an equation over a suitable effect.

\subsection{A monadic interface for black boxes}

We fix a type \(\Lambda\) of hidden variables (or \emph{ontic states}).
A preparation procedure samples a value of type \(\Lambda\), and the two
boxes are then evaluated relative to that value.  We model the local
behavior of a box using a monadic effect \(m\), allowing each box to be
probabilistic, nondeterministic, or otherwise effectful.

\begin{minted}{haskell}
type BB m = Bool -> Reader Lambda (m Bool)
\end{minted}

Intuitively, the \hc{Reader Lambda} layer represents whatever was fixed
at preparation time, while the effect \(m\) represents the remaining
local behavior (for example, probabilistic choice).

Given two boxes and a pair of settings, we can always form a joint
experiment producing a pair of outcomes:
\begin{minted}{haskell}
type Joint m = (BB m, BB m) -> (Bool,Bool) -> Reader Lambda (m (Bool,Bool))
\end{minted}

Different notions of locality correspond to different laws satisfied by
the joint evaluator.

\subsection{Canonical factorization and Bell locality}

The simplest joint evaluator combines the two local computations
independently.  In programming-language terms this is just applicative
pairing:

\begin{minted}{haskell}
evalA :: Applicative m => Joint m
evalA (bb1,bb2) (b1,b2) = liftA2 (,) (bb1 b1) (bb2 b2)
\end{minted}

When \(m\) is the distribution monad, \hc{evalA} expresses Bell’s
factorization condition:
\[
p(a,b \mid b_1,b_2,\lambda)
=
p(a \mid b_1,\lambda)\,p(b \mid b_2,\lambda).
\]
Bell’s theorem implies that any model satisfying this factorization,
together with the usual independence assumptions about preparation and
settings, has CHSH score bounded by \(2\).

\subsection{Locality principles}

We now collect the standard locality principles and express each as an
equation or constraint on an evaluator \hc{eval :: Joint m}.  When
\(m\) is probabilistic, these principles reduce to their usual
probabilistic formulations.

\paragraph{Single-valuedness.}
Single-valuedness says that, for fixed \(\lambda\) and a fixed setting,
a box produces a definite outcome rather than a nontrivial distribution.
Equivalently, each local computation is observationally equal to a
\hc{pure} value.  Writing \(\approx\) for observational equivalence in
the effect \(m\), we require:
\[
\forall\,bb, b.\;\;\exists\,v \in \{0,1\}.\;\; bb\,b \approx \texttt{pure}\;v.
\]
For the distribution monad, this says that the conditional distribution
is a Dirac measure.

\emph{Representative effect.}  \hc{Identity} (or Dirac distributions
inside a probabilistic monad).

\paragraph{Strong determinism.}
Strong determinism is single-valuedness together with strict locality of
dependence: each output is determined by its own setting and the shared
\(\lambda\).  In our interface this is captured by taking \(m\) to be
pure:
\[
\texttt{BB Identity} \;\cong\; \texttt{Bool -> Reader Lambda Bool}.
\]

\emph{Representative effect.}  \hc{Reader Lambda} (pure computation).

\paragraph{Weak determinism.}
Weak determinism does not demand such a strong separation.  It only
requires that, for fixed \(\lambda\), the \emph{pair} of outputs is
determined by the \emph{pair} of settings.  This allows the joint
behavior to depend on both settings without requiring a factorization
into two independent local response functions.  Equationally, weak
determinism says that the joint experiment is single-valued:
\[
\forall\,bb_1,bb_2,b_1,b_2.\;\;\exists\,v.\;\;
\texttt{eval (bb1,bb2) (b1,b2)} \approx \texttt{pure}\;v.
\]

\emph{Representative effect.}  \hc{Reader Lambda} with a pure joint
\hc{eval}, without requiring \hc{eval = evalA}.

\paragraph{Outcome independence.}
Outcome independence says that, for fixed \(\lambda\) and settings, the
two outcomes are produced independently.  In our semantics this is
exactly the requirement that the joint evaluator factorizes through the
canonical applicative pairing:
\[
\texttt{eval} \;=\; \texttt{evalA}.
\]
For probabilistic \(m\), this is precisely the statement that the joint
distribution is the product of the two local distributions (for fixed
\(\lambda\)).

\emph{Representative effect.}  The distribution monad with applicative
pairing.

\paragraph{Parameter independence.}
Parameter independence says that, once \(\lambda\) is fixed, the
distribution of Alice’s outcome does not depend on Bob’s setting, and
vice versa.  Using observational equivalence \(\approx\), we express this
as the pair of constraints
\[
\texttt{fmap fst (eval (bb1,bb2) (b1,b2))}
\approx
\texttt{fmap fst (eval (bb1,bb2) (b1,b2'))},
\]
\[
\texttt{fmap snd (eval (bb1,bb2) (b1,b2))}
\approx
\texttt{fmap snd (eval (bb1,bb2) (b1',b2))},
\]
for all settings \(b_1,b_1',b_2,b_2'\).  When \(m\) is probabilistic,
these are exactly the usual no-signalling equalities for the marginals.

\emph{Representative effect.}  A probabilistic effect equipped with a
no-signalling joint \hc{eval}.  Quantum correlations satisfy parameter
independence while violating outcome independence.

\paragraph{\(\lambda\)-independence (measurement independence).}
Finally, \(\lambda\)-independence is an assumption about preparation.  It
says that the hidden variable generated at preparation time is
independent of the later choice of settings.  In probabilistic terms,
this is the familiar condition \(p(\lambda \mid b_1,b_2)=p(\lambda)\).

In our semantics, \(\lambda\)-independence is enforced by staging: the
preparation procedure \hc{prep} runs before either box is given its
input, and the settings do not influence \hc{prep}.  The overall
experiment therefore has the shape
\[
\texttt{prep >>= \textbackslash lam -> runReader (eval (bb1,bb2) (b1,b2)) lam}.
\]

\emph{Representative effect.}  \hc{prep :: Dist Lambda} together with
\hc{Reader Lambda} for evaluation.

\subsection{From Bell locality to intermediate notions}

The classical local hidden-variable picture corresponds to
\(\lambda\)-independence together with outcome independence and
typically strong determinism.  Under these assumptions, the CHSH score
is bounded by \(2\).  At the opposite extreme, allowing general
interaction between the two local computations breaks parameter
independence and makes it easy to reach the algebraic maximum \(4\).

Quantum correlations sit in between.  They preserve parameter
independence (no signalling) but violate outcome independence
(factorization).  Operationally, this means that a semantics capable of
modeling quantum behavior must provide a joint evaluator \hc{eval} that
cannot be reduced to the canonical applicative pairing \hc{evalA}, while
still satisfying the no-signalling constraints above.

In the remainder of the paper we develop a semantics parameterized by an
effect, and we isolate the precise point at which classical applicative
factorization must be replaced by a more delicate composition principle.

%Need to discuss: outcome independence; weak determinism; strong
%determinism; parameter independence; $\lambda$-independence; and
%single valuedness; represent each by equations on a monad and find representative monad 

%=======================================================
\section{Empirical Model}
%=======================================================
Suppose there are two experimenters, Alice ($\textcolor{Bittersweet}{A}$) and Bob ($\textcolor{RoyalBlue}{B}$), each measuring a particle of a system. Both $A$ and $B$ have their own choices of measurements denoted as $I_A,I_B$, from which they can freely pick one to perform, and the associated sets of outcomes are denoted as $O_A, O_B$ respectively. Specifically, the Bell configuration~\cite{Bell1964} we adopt is the following one:
\begin{align*}
    \{A_1, A_2, A_3\} = I_A \ni \textcolor{Bittersweet}{A_i \mapsto a} \in O_A = \{\texttt{true}, \texttt{false}\},\\
    \{B_1, B_2, B_3\} = I_B \ni \textcolor{RoyalBlue}{B_j \mapsto b} \in O_B = \{\texttt{true}, \texttt{false}\},
\end{align*}
where $i,j\in\{1,2,3\}=I$ are indices. An \emph{empirical event} is represented as
\begin{align*}
    (A_i, B_j, a, b) \in  I_A \times I_B \times O_A \times O_B = \Psi. 
\end{align*}

Intuitively, an event is the abstraction of a “trial” in the experiment, where each part performs exactly one measurement upon a state of the system.

\begin{definition}
    An \emph{empirical model} is a pair $(\Psi, q)$ where $q$ is a probability measure on the event space $\Psi$~\footnote{The notations are modified from \cite{brandenburger_2008classification}.}, with
    \begin{align*}
        q( \textcolor{Bittersweet}{A_i \mapsto a} ) &\doteq q(a | A_i), \\
        q( \textcolor{RoyalBlue}{B_j \mapsto b}   ) &\doteq q(b|B_j), \\
        q( \textcolor{Bittersweet}{A_i \mapsto a} \land \textcolor{RoyalBlue}{B_j \mapsto b} ) &\doteq q(a,b|A_i,B_j). \\
    \end{align*}
\end{definition}

\begin{example}[\cite{Bell1964, mermin_1981quantum}]
With $q(A_i,B_j)=\frac{1}{9}$ for all $i,j$:
    \begin{equation}
        \tag{EM-Bell}
        q( \textcolor{Bittersweet}{A_i \mapsto a} \land \textcolor{RoyalBlue}{B_j \mapsto b} ) =
        \begin{cases}
            \frac{1}{2}, & \text{if} \ \ i=j \land a=b, \\
            0, & \text{if} \ \ i=j \land a \ne b, \\
            \frac{1}{8}, & \text{if} \ \ i \ne j \land a=b, \\
            \frac{3}{8}, & \text{if} \ \ i \ne j \land a\ne b.
        \end{cases}
        \label{eq:empirical-bell}
    \end{equation}
    Specifically, for each $i\in \{1,2,3\}$, $q( a = b | A_i, B_i) = 1 \land q(a\neq b | A_i,B_i)=0$.
\end{example}




%=======================================================
\section{Local Hidden-Variable Model}
%=======================================================
Let $\Lambda$ be a finite space, in which a “hidden variable” $\lambda$ dwells.

\begin{definition}
    A \emph{hidden-variable} model is a pair $(\Psi\times\Lambda,p)$ where $p$ is a probability measure on $\Psi\times\Lambda$, with
    \begin{align*}
        p_{\lambda}( \textcolor{Bittersweet}{A_i \mapsto a} ) &\doteq p(a | A_i, \lambda), \\
        p_{\lambda}( \textcolor{RoyalBlue}{B_j \mapsto b}   ) &\doteq p(b | B_j, \lambda), \\
        p_{\lambda}( \textcolor{Bittersweet}{A_i \mapsto a} \land \textcolor{RoyalBlue}{B_j \mapsto b} ) &\doteq  p(a,b | A_i, B_j, \lambda).
    \end{align*}
    For simplicity, we assume $p(\lambda)>0$ for each $\lambda\in\Lambda$.
\end{definition}
In addition to the definition, there are two major assumptions of a \emph{local} hidden-variable theory, namely \emph{$\lambda$-Independence}~\eqref{eq:lambda-idependence} and \emph{Principle of Locality}~\eqref{eq:locality}. As we will see, both are desirable properties concerning $\Lambda$, which are expected in a very fundamental way, by almost all scientific experiments.

%=======================================================
\subsection{$\lambda$-Independence}

For each $\lambda \in \Lambda$, and any $A_i\in I_A, B_j\in I_B$, we have
\begin{equation}
    \tag{$\lambda$\textbf{I}}
    p(\lambda) = p(\lambda | A_i) = p(\lambda | B_j) = p(\lambda | A_i, B_j)
    \label{eq:lambda-idependence}
\end{equation}
which ensures that $\lambda$ is independent of the measurement choices of both $A$ and $B$. 

The intuition behind~\eqref{eq:lambda-idependence} is that, the intrinsic properties of particles (the observed), “things-in-themselves” which do not appear to the observers, as represented by $\lambda\in\Lambda$, are independent of the observers' measurement choices~\footnote{Conversely, the assumptions like $p(A_i,B_j) = p(A_i,B_j | \lambda )$  emphasize the observers' \emph{freedom of choice}. They are related by Bayes' theorem $p(\lambda | A_i,B_j) \cdot p(A_i,B_j) = p(A_i,B_j|\lambda) \cdot p(\lambda)$. }.

%=======================================================
\subsection{Principle of Locality}

For each $\lambda\in \Lambda$, and $A_i\in I_A, B_j\in I_B$, $a\in O_A, b\in O_B$, we have
\begin{equation}
    \tag{\textbf{L}}
    p_{\lambda}( \textcolor{Bittersweet}{A_i \mapsto a} \land \textcolor{RoyalBlue}{B_j \mapsto b} ) 
        = p_{\lambda}( \textcolor{Bittersweet}{A_i \mapsto a} ) \times p_{\lambda}( \textcolor{RoyalBlue}{B_j \mapsto b} )
    \label{eq:locality}
\end{equation}
which ensures that the outcome $a \in O_A$ is independent of the measurement choice $B_j\in I_B$ \emph{and} the outcome of $b \in O_B$, and vice versa~\footnote{The alternative name for~\eqref{eq:locality} is \emph{factorizability}.}.

The physical intuition of~\eqref{eq:locality} is that once conditioned by $\lambda\in\Lambda$, the measurements/outcomes of two experimenters $A$ and $B$ are “causally separated”. That is,
\[
    A_i \leadsto a \flowsfromb \fbox{$\lambda$} \leadsto b \flowsfromb B_j
\]
where arrows $\leadsto,\flowsfromb$ represent “causal flow” and read as “influence” or “cause”.




%=======================================================
\section{Bell's Theorem}
%=======================================================
\begin{theorem}[\cite{Bell1964}]
    For the empirical model $(\Psi,q)$ defined as~\eqref{eq:empirical-bell}, there is no equivalent hidden-variable model $(\Psi\times \Lambda, p)$ satisfying both~\eqref{eq:lambda-idependence} and~\eqref{eq:locality}.
\end{theorem}

\begin{proof}[\normalfont\scshape  Proof]
    Assume such a hidden-variable model $(\Psi\times\Lambda,p)$ satisfying ~\eqref{eq:lambda-idependence} and~\eqref{eq:locality}.
    \begin{align*}
        q(\textcolor{Bittersweet}{A_i \mapsto a} \land \textcolor{RoyalBlue}{B_j \mapsto b} ) 
            &= \sum_{\lambda\in\Lambda}^{} \ p_{\lambda}(\textcolor{Bittersweet}{A_i \mapsto a} \land \textcolor{RoyalBlue}{B_j \mapsto b}) \times p(\lambda | A_i, B_j) \\
            &= \sum_{\lambda\in\Lambda}^{} \ p(\lambda) \times p_{\lambda}(\textcolor{Bittersweet}{A_i \mapsto a} \land \textcolor{RoyalBlue}{B_j \mapsto b}) && \text{by}~\eqref{eq:lambda-idependence} \\
            &= \sum_{\lambda\in\Lambda}^{} \ p(\lambda) \times p_{\lambda}(\textcolor{Bittersweet}{A_i \mapsto a}) \times p_{\lambda}(\textcolor{RoyalBlue}{B_j \mapsto b}). &&\text{by} ~\eqref{eq:locality}
    \end{align*}
    Given any $\lambda$, consider $i=j\in\{1,2,3\}$. With~\eqref{eq:empirical-bell}, we have each of
    \begin{align*}
        p_{\lambda}(\textcolor{Bittersweet}{A_i \mapsto \texttt{true}}) \times p_{\lambda}(\textcolor{RoyalBlue}{B_i \mapsto \texttt{false}}) &= 0, \text{and} \\
        p_{\lambda}(\textcolor{Bittersweet}{A_i \mapsto \texttt{false}}) \times p_{\lambda}(\textcolor{RoyalBlue}{B_i \mapsto \texttt{true}}) &= 0,
    \end{align*}
    which says $\neg (A_i \oplus B_i)$. Since $\sum_{}^{}{p_{\lambda}(A_i\mapsto a)=\sum_{}{p_\lambda(B_i\mapsto b)=1}}$, either
    \begin{align*}
        p_{\lambda}(\textcolor{Bittersweet}{A_i \mapsto \texttt{true}}) =1 \land   p_{\lambda}(\textcolor{RoyalBlue}{B_i \mapsto \texttt{true}}) = 1, \text{or} \\
        p_{\lambda}(\textcolor{Bittersweet}{A_i \mapsto \texttt{false}}) =1 \land  p_{\lambda}(\textcolor{RoyalBlue}{B_i \mapsto \texttt{false}}) = 1,
    \end{align*}
    which says $A_i \leftrightarrow B_i$. This implies \emph{determinism}~\cite{abramsky_2013relational}: we can define
    \begin{align*}
        \vec{\Delta} &: \Lambda \rightarrow \{\texttt{true}, \texttt{false}\}^3 \\
        \vec{\Delta}(\lambda) &= (\Delta_1(\lambda), \Delta_2(\lambda), \Delta_3(\lambda))
    \end{align*}
    where $\Delta_i(\lambda) \in \{\texttt{true}, \texttt{false}\}$ \emph{determines} an outcome for both $A_i,B_i$ given $i$ and $\lambda$~\footnote{$\vec{\Delta}$ can be viewed as a conjunction of three different ways to partition/characterize $\Lambda$ into two disjoint subsets, or simply three \emph{properties} of $\lambda$. $\vec{\Delta}(\lambda)\in \{\texttt{true}, \texttt{false}\}^3$ corresponds to Mermin's deterministic “instruction sets”~\cite{mermin_1981quantum}, which is not \emph{assumed} for the proof but \emph{derived} from the proof.}.
    
    Therefore, we can rewrite
    \begin{align*}
        q(\textcolor{Bittersweet}{A_i \mapsto a} \land \textcolor{RoyalBlue}{B_j \mapsto b} ) &= \sum_{S_{ij}=\{\lambda\in\Lambda \mid \Delta_i(\lambda)=a\land\Delta_j(\lambda)=b   \}}{p(\lambda)}.
    \end{align*}
    
    Focus on the following three cases in~\eqref{eq:empirical-bell}:
    \begin{align*}
        q(\textcolor{Bittersweet}{A_1 \mapsto \texttt{true}} \land \textcolor{RoyalBlue}{B_2 \mapsto \texttt{false}} ) &=3/8, & S_{12}=\{\lambda\in\Lambda \mid \Delta_1(\lambda) \land \neg \Delta_2(\lambda)\}, \\ 
        q(\textcolor{Bittersweet}{A_2 \mapsto \texttt{true}} \land \textcolor{RoyalBlue}{B_3 \mapsto \texttt{false}} ) &=3/8, & S_{23}=\{\lambda\in\Lambda \mid \Delta_2(\lambda) \land \neg \Delta_3(\lambda)\}, \\
        q(\textcolor{Bittersweet}{A_3 \mapsto \texttt{true}} \land \textcolor{RoyalBlue}{B_1 \mapsto \texttt{false}} ) &=3/8, & S_{31}=\{\lambda\in\Lambda \mid \Delta_3(\lambda) \land \neg \Delta_1(\lambda)\},
    \end{align*}
    where $S_{12},S_{23},S_{31}$ are \emph{pairwise disjoint}. We can derive an inequality:
    \begin{align*}
        9/8 =\sum_{S_{12}}{p(\lambda)} + \sum_{S_{23}}{p(\lambda)} + \sum_{S_{31}}{p(\lambda)} = \sum_{S_{12}\cup S_{23}\cup S_{31}}{p(\lambda)} \leq \sum_{\Lambda}{p(\lambda)} =1
    \end{align*}
    witnessing a contradiction.
\end{proof}

%=======================================================
\section{Effectful Hidden-Variable Model}
%=======================================================

Starting from the determination function $\vec{\Delta}:\Lambda\rightarrow\{\texttt{true},\texttt{false}\}^3$ defined in the proof, we can naturally derive a definition of “proto-qubit”.

We noticed that, the function $\Delta_i:\Lambda\rightarrow\{\texttt{true},\texttt{false}\}$ is \emph{pure}. This indicates the space of hidden-variables $\Lambda$ has 3 predefined, “intrinsic properties”.



\begin{comment}
We can define \emph{co-separator} of $\Lambda$:
\begin{equation}
    (\lambda_1 = \lambda_2) \overset{\mathrm{def}}{=} \forall (i\in I). \forall (\Delta_i:\Lambda\rightarrow\{\texttt{true,\texttt{false}}\}). [ \ \Delta_i(\lambda_1) \leftrightarrow \Delta_i(\lambda_2) \ ].
\end{equation}



\begin{minted}[mathescape]{haskell}
    qa  :: Dist |$ I_A $|
    qb  :: Dist |$ I_B $|
    qab :: Dist (|$I_A$|, |$I_B$|)
\end{minted}


\begin{minted}[mathescape]{haskell}
    emA :: |$ I_A \rightarrow $| Dist Bool
    emB :: |$ I_B \rightarrow $| Dist Bool
    em  :: |$ I_A \rightarrow I_B \rightarrow $| Dist (Bool, Bool)
\end{minted}

\begin{minted}[mathescape]{haskell}
    uniformLam :: Dist |$\Lambda$|
\end{minted}


\begin{minted}[mathescape]{haskell}
    d1, d2, d3 :: |$ \Lambda \rightarrow $| Bool
    d :: |$ \Lambda \rightarrow $| (Bool, Bool, Bool)
\end{minted}
\end{comment}


%=======================================================
\section{Relating to Interpretations of QM}

There are nonetheless several implicit assumptions in the proof of Bell's theorem.

\subsection{Superdeterminism}

\[
    p(\lambda | C_1) \neq p(\lambda|C_2) 
\]
for some $C_1, C_2$.

\subsection{Nonlocal hidden-variables}

\subsubsection{Bohmian Mechanics}

Space.

\subsubsection{Transaction Interpretation}

Time.

%%%%%%%%%%%%%%%%%%%%%%%%%%%%%%%%%%%%%%%%%%%%%%%%%%%%%%%%%%%%%%%%%%%%%%%

\bibliography{cites}
\end{document}
